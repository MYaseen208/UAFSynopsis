\section{Review of Literature}
A crop model is based on mathematical representation of a canopy. Such models may be built for a multiplicity of purposes they may be differing extensively in complexity, scope and focus. This has led to a wide range of models simulating particular aspects of the processes and particular crops involved in plant growth and development to be created. Empirical model is the type of model in which no knowledge of underlying processes is involved. Purely mechanistic modeling which incorporate knowledge of intermediate between semi-empirical models and processes acting within the system.

 Crop models also subdivided in terms of the basic units modeled. Model the canopy as a homogeneous medium with state variables representing spatial averages of concentration such as Leaf Area Index (LAI) which is leaf area of one-sided per unit ground or biomass. In recent times, advances in computing power have made possible the modeling of the canopy as a population of individually modeled plants, which in turn may be modeled as an ensemble of individual organs. In theory modeling at the plant level should be accurate and satisfying since many important processes such as assimilation occur at the level of the individual plant, however such models tend to require heavy parameterization and thus present their own particular problems.

 \citet{Fisher2000} explained crop modeling area covers several plants and trees of interest, from fowers to financial crops,  rice \citet{JameCutforth1996} and maize. Models can also be presented which in genetics of the plant development, growth and also root structure and development.


\subsection{Empirical Model and Semi-Empirical Models} 
 \citet{Marcelis1998} studied the crop-growth models are mathematical functions, or Statistical relationships such as exponential functions, polynomial's and sigmoidal curves representing the state of the canopy as a function of time. Statistical models involve wide-ranging data collection, growing season of the crop of interest and over many years. These models are essentially limited in their application and genotype from which data was collected to create the models. The predictive value of descriptive models can be in elevation, because obliquely take into account all the unknown affects as as well.  Empirical models are invented in statistical relationships between a variable of interest such as biomass or LAL and time. Empirical models have little heuristic value but can produce good predictions, especially in the environmental conditions for which the models are applied in within the range of variation which the model is parameterized. Empirical models are used in substantial re-calibration. Now empirical elements in models are still common in many mechanistic models.

 \citet{Bloomenthal1985} suggested that  empirical models of crop structure require data on the plant's geometric features. The architecture of a plant plays a fundamental role in the allocation of resources acquisition, tolerance to damage and competition. Such models incorporate the geometry and structure that are useful tools for plant scientists and in ecology, biology, agronomy, pest management and remote sensing.

  \citet{Fournier2003} demonstrated the order to increase their ability to be applied in different locations and the generality of these models and they must encompass more knowledge of the processes involved. These models are known as semi-empirical models and an example of dynamic empirical model. This is computed by temperature instead of time and observations based on growth rate are constant within limited range of temperature. Dynamic model expressing growth and biomass as a differential equation. The behavior of the system is obtained through integration of the model. 

 \citet{Waggoner1984} suggested that additional variables can also be included in the model to increase its generalization, which predicts wheat yield is a function of meteorological variables, such as precipitation number of days warmer than $32^{\circ}\text{C}$ temperature. In Presence of extra variables within the model aid in increasing and its applicability as it can be more easily re-calibrated of areas other than those where the data was collected to create the model.

 \citet{Prince1991} examined modular structure of  PEM has allowed it to be modified by other researchers, included different stress factors which enables the maximum efficiency from departure and caused a physiological reactions to restrictive environmental conditions. Another improvement has included making the use of efficiency in function of water, nutrient stress, temperature and joining the models. PEM is used to estimate global network net primary production (NPP), global carbon cycle, and at regional scale of crop production.

  \citet{Disney2000}  recongnized  various options for MC Ray Tracing in canopy applications. These models also used in detailed architecture of the canopy structure and calculate the intersections of rays fired into the 3D scene. The objects are determined whether the photons are observed or scattered at each intersection this method can be used static empirical models of the canopy. Large computational times are associated with this model especially when diffuse scattering is simulated and combined AFRC wheat and sail or found that only up updating the model later on in development within the use of remote sensing data that Leaf Area Index prediction was improved. This is not a great value of farmers at this late stage in development and no methods exist to aid in increasing the yield if it predicted to reduce. However, it can help to produce a more accurate yield map which is of used in precision farming.  
\subsection{Effect of Nitrogen and Cultivars} 
 \citet{Costa2002} evaluated effect of nitrogen rates on maize genotypes. The genotypes were NLRS, LNS, LRS and conventional hybrid and late maturity. The genotype consistently yielded 12.39 and 10.29 in 1997 and 1998 respectively, while NLRS performed not well, however genotype yield grain placing varies among sites. In General leafy reduced stature out yielded its conservative counter part by 26\% at the one side and 12\% at other.     

  \citet{Oikeh2003} presented N differential equations of maize cultivars in West Africa under N fertilization and stated that TZB-SR cultivars composed additional N in above ground parts of plant, both years as related to the other cultivars. All, excepting at the time of slinking the (SPL) semi-prolific late variety, happened about 50-60 \% of their N requirements. In both years, SPL had the maximum grain concentration and least deceptive of N loss over leaching which has greater capacity to take up N through the grain filing period in the next year.They determined that the use of maize with high N acceptance capacity during the grain filling period.

 \citet{DAndrea2006} conducted a study in Argentina to analyze the response of N different availability of morph physiology traits in a set of 12 maize ingrained lines, from  breeding eras and different origins (Argentina and USA). Traits involved in the analysis were related to yield components, shoot biomass production, canopy structure, grain yield and light interception. They concluded that difference for these parameters were significant among genotypes. \citet{Andrade1993} studied relationship between intercepted radiation, kernel number per umit area in maize and flowering found a significant positive association and 5.9 kernels radiation utilization were obtained under shading experiments.

 \citet{Kiniry2001} suggested that simulation of crop development, yield and growth accomplished through evaluating the growth rate, the stage of crop development and partitioning of biomass into growing organs. All of these processes are affected by environmental and cultivar specific factors. These processes are dynamic. The detailed description of key processes in crop provide help to system of simulating grain yield production and provides a means of quantifying how cultivars differ using crop model.     

  \citet{Ogunlela1988} conducted an experiment on yield component of field grown by using nitrogen fertilization ranging 50 to 200 kg. They estimated the kernel depth and plant height and described how whole plant should be simulated at the organ smooth and how crop models should simulate processes at the whole plant level.
\subsection{Maize Production and Cimate Change}
  \citet{Parry2000} explained climate change affect agriculture field in different parts of the world. The properties among various continents are influenced by on soil conditions, availability of resources current climatic conditions and infrastructure use to crop with change. These differences are also probable to greatly influence the reaction to climatic change.     

 \citet{Tao2006} studied the developments in phenology, climate change and yield of crops (rice, wheat, maize) in China. Some Significant results tend to observed at most of the investigated areas in which during the two decades some results were changes in temperature had shifted crop phenology and affected crop yields. The study highlighted the concentration on physiological processes, further investigation showed that the mutual impacts of temperature and mechanisms growth.  

 \citet{Ritchie1998} studied the cereal crop growth, yield and development included Decision Support System for Agro-technology (DSSAT) using CERES Crop simulation model. Results were obtained to show that when the cultivar, weather and management information are realistically quantified, the yield results are usually within acceptable limits.

 \citet{Probert2001} collected climate data of seven sites and used as input for the maize model CMKEN to explore a number of management options that impinge on maize yields and reported that the cultivar Katumni composite B is well adapted for whole region as compared to other cultivars. They further concluded that yield potential is strongly dependent on rainfall regime and soil type, nitrogen rates also vary with rainfall regime.

