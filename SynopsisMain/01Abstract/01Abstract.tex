\begin{center}
\textbf{\underline{\LARGE{\textsc{University of Agriculture, Faisalabad}}}}\\

\vspace{3mm}
\textbf{\large{\textsc{Department of Mathematics and Statistics}}}\\

\vspace{3mm}
(Synopsis for MPhil degree in Statistics)
\end{center}

\vspace*{-8mm}

\textmd{
\large{
\begin{description}
\item[TITLE:] \textbf{Evaluating the impact of climate changes on maize productivity in Punjab using dynamic crop models}
\end{description} 
}
}

\vspace{3mm}

\hspace{1cm}
\begin{tabular}{lcl}
Name of the Student & : \hspace{2.5cm} & Name of Student\\
Registration Number & : \hspace{2.5cm} & Reg. No of Student
\end{tabular}


\vspace{5mm}


\begin{abstract}
Agriculture is highly dependent on climate and therefore, changes in global climate could have major effects on crop yield and thus food supply. The future prosperity and economic stability of Pakistan mainly depend upon the material resources and utilization. Therefore, there is a dire need for advanced planning to increase food production and improve quality to meet the needs of increasing population. Dynamic crop modeling is a technique which is used in crop production. The essential characteristic of simulation method is reproduced in model, which is studied in different time scale. Crop models are mathematical models which described the growth and development of a crop interacting with soil. In Pakistan maize is the fourth largest grown crop after wheat, cotton and rice. In this study maize crop data will be used. This modeling will be used to estimate the agriculture maize production as a function of soil and weather as well as crop management. The study will computed state variables rate over time, from planting until harvest maturity or final harvest and estimate the parameters of dynamic model such as excitation coefficient, radiation use efficiency, maximum leaf area index, analysis the effect of nitrogen and cultivars are expected to evaluate the capability of maize model growth and yield of different locations in Punjab.
\end{abstract}