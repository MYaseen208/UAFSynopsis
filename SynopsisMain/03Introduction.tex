\section{Introduction}
Crop models have been established and used in wide-range as operational and decision maintenance tools in resource management system or crop production. Crops models at their simplest yield and at their most complexes simulate the processes involved in crop growth and development. Crop models are also used to estimate the effect of environment change on crop production as a consequence of increasing greenhouse gasses. All crop models, mechanistic or empirical require data which can be collected using measurements or the trawling through in the case of regional yield estimates. Crop models are mathematical models which described the growth and development of a crop interacting with soil. There are two forms of crop modeling dynamic crop modeling and response crop modeling. Response crop model is set of equations for responses of interest as function of explanatory variables. \citet{Loomis1968} discussed the mechanistic of models in simulating cropping structures at one level that described by processes at a low level.

  Dynamic system models are generally used in extension or agronomic research. These models represented in differential or difference equations that represent the dynamics of the different components of the system (plant, soil etc.) and can also be used to explore the effects that caused changes in the environments.
Maize is maximum yielding cereal crop in the world. It has a significant importance in Pakistan where rapidly increase in population has already out exposed the available in food supplies. In ranking maize is in fourth most grown crop in the world with an area of more 118 million hectares with an annual production of about 600 million metric tons. The expanded use of maize in industry gives this crop a prominent place in agricultural economy. In Pakistan, maize is the fourth largest grown crop after wheat, cotton and rice \citep{RasheedAliMahmood2004}. The area under maize is over one million hectares and production 3.5 million metric tons. The area cover maize is over one million hectares and its production is 3.5 million metric tons. Punjab contributes 39\% area; KPK contributes 56\% area and 3\% of the area contributed by Sindh and Baluchistan.  The future economic stability and prosperity of Pakistan mainly depends upon the quantum of material resource, utilization, and judicious explanation.

 Maize grows from sea level to 3000 meters and it is a warm weather plant. Maize requires extensive moisture and warm weather from germination to flowering. The most appropriate temperature for germination is $21^{\circ}\text{C}$ and for growth $32^{\circ}\text{C}$ . It also can be grown under different condition. Monsoon season gives the highest production of maize. Extreme high temperature and low humidity during flowering damage the underground and interfere with proper pollination, resulting in poor grain formation. Maize is very sensitive to stagnant water especially during its early stages of growth. The elements of dynamic crop model are state variables and explanatory variables. These variables play an important role in the crop modeling. State Variables describe the conditions of system components such as change with time in dynamic models as system components interact with each other and the environment such as Soil water content, crop biomass, leaf area index, and plan emergence time. 
 
 \citet{Sinclair1996}  suggested that crop dynamic models have been developed as a creative tool for agronomic management strategy evaluation. Various fields have used these models for decision making in agriculture field and studied the relationship among management, environment yield variability. 
\subsection{Objectives}
 The objectives of this study are : 
\begin{itemize}
	
\item Simulate a dynamic crop model for maize and make modification to the maize model for improving the ability of the model to predict response to weather at different locations of Punjab.  
\item	Estimate the parameters of complete dynamic maize model. 
\item  Evaluate the effect of cultivars and nitrogen rates on growth of maize. 
  \end{itemize}